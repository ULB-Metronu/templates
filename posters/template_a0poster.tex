%\title{LaTeX Portrait Poster Template}
%%%%%%%%%%%%%%%%%%%%%%%%%%%%%%%%%%%%%%%%%
% a0poster Portrait Poster
% LaTeX Template
% Version 1.0 (22/06/13)
%
% The a0poster class was created by:
% Gerlinde Kettl and Matthias Weiser (tex@kettl.de)
%
% This template has been downloaded from:
% http://www.LaTeXTemplates.com
%
% License:
% CC BY-NC-SA 3.0 (http://creativecommons.org/licenses/by-nc-sa/3.0/)
%
%%%%%%%%%%%%%%%%%%%%%%%%%%%%%%%%%%%%%%%%%

%----------------------------------------------------------------------------------------
%	PACKAGES AND OTHER DOCUMENT CONFIGURATIONS
%----------------------------------------------------------------------------------------

\documentclass[a0,portrait]{a0poster}

\usepackage{multicol} % This is so we can have multiple columns of text side-by-side
\columnsep=100pt % This is the amount of white space between the columns in the poster
\columnseprule=3pt % This is the thickness of the black line between the columns in the poster

\usepackage[svgnames]{xcolor} % Specify colors by their 'svgnames', for a full list of all colors available see here: http://www.latextemplates.com/svgnames-colors

\usepackage{times} % Use the times font
%\usepackage{palatino} % Uncomment to use the Palatino font
\usepackage[utf8]{inputenc}
\usepackage[USenglish]{babel}

\usepackage{graphicx} % Required for including images
\graphicspath{{figures/}} % Location of the graphics files
\usepackage{booktabs} % Top and bottom rules for table
\usepackage[font=small,labelfont=bf]{caption} % Required for specifying captions to tables and figures
\usepackage{amsfonts, amsmath, amsthm, amssymb} % For math fonts, symbols and environments
\usepackage{wrapfig} % Allows wrapping text around tables and figures
\usepackage{float}
\usepackage{subfig}
\usepackage{siunitx}

\definecolor{britishracinggreen}{rgb}{0.0, 0.26, 0.15}

\begin{document}

%----------------------------------------------------------------------------------------
%	POSTER HEADER
%----------------------------------------------------------------------------------------

% The header is divided into two boxes:
% The first is 75% wide and houses the title, subtitle, names, university/organization and contact information
% The second is 25% wide and houses a logo for your university/organization or a photo of you
% The widths of these boxes can be easily edited to accommodate your content as you see fit

\VeryHuge \color{black} \begin{center}
\textbf{Title} % Title
\end{center}
\vspace{0cm} % A bit of extra whitespace between the header and poster content

\begin{minipage}[b]{0.6\linewidth}
\Large \textbf{Authors \\
Affiliations \\}\\
\textsuperscript{1}Second affiliation  \\[0.5cm] % Author(s)
\Large \texttt{mail@ulb.be}
\end{minipage}
%
\begin{minipage}[b]{0.4\linewidth}
\hspace*{2cm}
\includegraphics[height=6cm]{logos/logo_ulb.png}\
\hspace*{3cm}
\includegraphics[height=6cm]{logos/qr-code.eps}\\[1cm]
\hspace*{2cm}
\end{minipage}

\vspace{-2cm} % A bit of extra whitespace between the header and poster content

%----------------------------------------------------------------------------------------

\begin{multicols}{2} % This is how many columns your poster will be broken into, a portrait poster is generally split into 2 columns
\normalsize
%----------------------------------------------------------------------------------------
%	INTRODUCTION
%----------------------------------------------------------------------------------------
\color{Navy} % DarkSlateGray color for the rest of the content
\color{Black}
Intro

%----------------------------------------------------------------------------------------
%	Tools and methods
%----------------------------------------------------------------------------------------

\color{Navy} % DarkSlateGray color for the rest of the content
\section{Validation of the model}
\color{Black}


\color{Navy} % DarkSlateGray color for the rest of the content
\section{Integration of realistic geometries}
\color{Black}


%----------------------------------------------------------------------------------------
%	CONCLUSIONS
%----------------------------------------------------------------------------------------

\color{Navy} % DarkSlateGray color for the rest of the content
\section*{Conclusions and outlook}
\color{Black}
Conclusions
%----------------------------------------------------------------------------------------
%	ACKNOWLEDGMENTS
%---------------------------------------------------------------------------------------
\color{Navy}
\section*{Acknowledgments}
\color{Black}
Acknowledgments
%----------------------------------------------------------------------------------------
%	REFERENCES
%---------------------------------------------------------------------------------------
\begin{small}
\color{britishracinggreen}
\begin{thebibliography}{12} % Use for 1-9 references

\bibitem{paper}
Author \emph{et al.}. \textquotedblleft{Title}\textquotedblright

\hphantom{2pt}
\end{thebibliography}
\end{small}
%----------------------------------------------------------------------------------------
\vfill\null
\columnbreak

%----------------------------------------------------------------------------------------
%	FIGURES
%----------------------------------------------------------------------------------------
\color{Brown}
\subsection*{BDSIM model of the IBA Proteus\textsuperscript{\textregistered} One system}
\color{Black}
\begin{center}\vspace{1cm}
\captionof{figure}{Caption.}
\label{fig:fig1}
\end{center}%\vspace{1cm}

%----------------------------------------------------------------------------------------


\end{multicols}
\end{document}
